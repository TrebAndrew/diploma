\chapter*{Введение}							% Заголовок
\addcontentsline{toc}{chapter}{Введение}	% Добавляем его в оглавление

\newcommand{\actuality}{}
\newcommand{\aim}{\textbf{Целью}}
\newcommand{\tasks}{задачи}
\newcommand{\defpositions}{\textbf{Основные положения, выносимые на~защиту:}}
\newcommand{\novelty}{\textbf{Научная новизна}}
\newcommand{\influence}{\textbf{Научная и практическая значимость}}
\newcommand{\reliability}{\textbf{Степень достоверности}}
\newcommand{\probation}{\textbf{Апробация работы.}}
\newcommand{\contribution}{\textbf{Личный вклад.}}
\newcommand{\publications}{\textbf{Публикации.}}

%{\actuality}
В настоящее время в Институте ядерной физики им. Г. И. Будкера СО РАН проводятся эксперименты на электрон-позитронном накопителе \text{ВЭПП-2000} \cite{VEPP} в диапозоне энергий от 320 МэВ до 2 ГэВ в системе центра масс. В местах встречи пучков установлены два детектора: криогенный магнитный детектор \text{КМД-3 \cite{CMD}} и сферический нейтральный детектор \text{СНД \cite{SND}.} Физическая программа экспериментов включает в себя измерение сечений электрон-позитронной аннигиляции в адроны на низких энергиях, а также измерение адронного вклада в вычисление аномального магнитного момента мюона {$(g-2)_\mu$}. Так как в данной области энергии расчеты в рамках пертурбативной квантовой хромодинамики невозможны, то основным источником изучения взаимодействия легких кварков являются данные, полученные в экспериментах. Набор экспериментальных данных с детектором \text{КМД-3} начат в декабре 2010 года. В апреле 2013 года эксперименты были остановлены в связи с модернизацией инжекционного комплекса \cite{status}. В конце 2016 года запущен новый инжекционный комплекс \text{ВЭПП-5 \cite{VEPP-5}} и возобновлен набор статистики детектором \text{КМД-3.}

 \aim\ данной работы является измерение сечения процесса {$e^+\:e^- \to K_{S}\:K_{L}\:\pi^0$}
 в области энергий от порога рождения \text{(1130 МэВ)} до \text{2 ГэВ.} В анализе процесса использовались данные, полученные в экспериментах 2011-2012 годов с детектором КМД-3. Интеграл светимости составил \text{$33.18$ пб$^{-1}$} в диапозоне энергий от \text{1.1 ГэВ} до \text{2 ГэВ. }
 %Процесс примечателен тем, что все три конечные частицы псевдоскалярные и имеют общую плоскость распада. Одно из главных физических свойства процесса - это нейтральность конечных частиц. (Ниже воткнуть.)
 %пересчитать точнее светимость для твоей области

В задачи данной работы входило:
\begin{enumerate}
  \item Выработать оптимальные критерии отбора событий для процесса {$e^+\:e^- \to K_{S}\:K_{L}\:\pi^0$}.
  \item Определить эффективность регистрации процесса с помощью Монте-Карло моделирования.
  \item Получить предварительное сечение процесса {$e^+\:e^- \to K_{S}\:K_{L}\:\pi^0$} в диапозоне энергий от порога рождения \text{(1130 МэВ)} до \text{2 ГэВ.}
\end{enumerate}

%\novelty\ данной работы заключается в исследовании малоизученного процесса  {e^+\:e^- \to $\itshape K_{S}\:K_{L}\:\pi^0$}, а так же измерение его вклада в поляризацию ваакума, используемое для вычисления аномального магнитного момента мюона {$(g-2)_\mu$}.

Сечение данного процесса до недавнего времени не было измерено, в настоящее время существуют пока что только предварительные результаты, представленные коллаборацией SND \cite{korneev}, и недавно опубликованная статья коллаборации BaBar \cite{BaBar}. %Результаты ранее сказанных коллабораций имеют различия в области энергий выше 1.8 ГэВ, таким образом, имеет место дополнительная задача --- подтвердить результаты одной из коллабораций.
%данной работы заключается в вычислении сечения процесса  {e^+\:e^- \to $\itshape K_{S}\:K_{L}\:\pi^0$} в настоящее время предварительно изммеренного коллаборациями BaBar и SND, а так же в измерении его вклада в поляризацию ваакума, используемое для вычисления аномального магнитного момента мюона {$(g-2)_\mu$}. %коряво

%Научная новизна заключается в том, что сечение данного процесса до сих пор не был измерен, в настоящее время существуют пока что только предварительные результаты, представленные коллаборациями Бабар и СНД.
%Сказать про физику здесь, изотопический спин, все частицы в одной плоскости, три псевдоскалярные частицы. % Характеристика работы по структуре во введении и в автореферате не отличается (ГОСТ Р 7.0.11, пункты 5.3.1 и 9.2.1), потому её загружаем из одного и того же внешнего файла, предварительно задав форму выделения некоторым параметрам

%% регистрируем счётчики в системе totcounter
\regtotcounter{totalcount@figure}
\regtotcounter{totalcount@table}       % Если поставить в преамбуле то ошибка в числе таблиц
\regtotcounter{TotPages}               % Если поставить в преамбуле то ошибка в числе страниц

%% на случай ошибок оставляю исходный кусок на месте, закомментированным
%Полный объём диссертации составляет  \ref*{TotPages}~страницу с~\totalfigures{}~рисунками и~\totaltables{}~таблицами. Список литературы содержит \total{citenum}~наименований.
%
Представленная работа посвящена разработке методов моделирования процесса генерации синхротронного излучения (СИ) от электронного пучка с конечным эмиттансом и прохождения этого излучения через оптическую систему. Развитие магнитных схем циклических ускорителей дало возможность снизить эмиттанс электронного пучка и приблизить источники СИ к дифракционному пределу для широкого диапазона длин волн, вплоть до жёсткого рентгена. Под дифракционным пределом мы понимаем, что эмиттанс электронного пучка $\epsilon_{x, y}$ много больше или, по крайней мерее, сравним с "эмиттансом" излучения -- $\lambda/4\pi$, то есть  $\epsilon_{x, y} \ll \lambda/4\pi$. Такое излучение характиризуется заметной степенью поперечной когерентностью. Случай с частичной когерентностью представляет наибольший интерес, так как именно он реализуется в большинстве практических случаях. В работе предложен оригинальный метод генерации частично когерентного синхротронного излучения и рассмотрены практические примеры распространения частично когерентного волнового фронта через оптическую систему источников СИ.

\rr{...}


