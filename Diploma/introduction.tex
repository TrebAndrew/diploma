\chapter*{Аннотация }							% Заголовок
Представленная работа посвящена разработке алгоритма моделирования частично когерентного синхротронного излучения и его апробации на примере нескольких рентгенооптических оптических схем. В работе рассматриваются теоретические основы процесса генерации и распространения частично когерентного синхротронного излучения, а так же его статистические характеристики. По ходу изложения приводится описание двух известных методов моделирования частично когерентного излучения, первый из которых обсуждался в литературе, однако не был описан в программном коде и, соответственно, не применялся при расчёте рентгенооптических схем синхротронных источников излучения. С помощью этого метода в работе описан эффект влияния продольной когерентности излучения на его угловую расходимость. При этом второй метод широко используется, не смотря на то, что обладает рядом недостатков, обсуждаемых в работе. Оба метода дают физически достоверный результат для наблюдаемых интенсивностей синхротронного излучения, однако только первый метод даёт физически правильные поля излучения. На основе проведённой работы, был создан алгоритм, называемый СЕРВАЛ, качественно отличающийся от уже известных подходов. Была проведена перекрёстная проверка СЕРВАЛа на совпадение генерируемого поля в источнике и дальней зоне, а также соответствующих функций взаимной когерентности. Применение СЕРВАЛ показано на примере трёх оптических схем: фокусировка при наличии конечной входной апертуры, классический двухщелевой эксперимент (интерферометр Юнга) и фокусировка рентгеновским зеркалом при наличии шероховатостей. В работе проанализированы границы применимости СЕРВАЛа и описано его быстродействие.
\chapter*{Введение}							% Заголовок
\addcontentsline{toc}{chapter}{Введение}	% Добавляем его в оглавление

\newcommand{\actuality}{Актуальность}
\newcommand{\aim}{\textbf{Целью}}
\newcommand{\tasks}{Задачи}
\newcommand{\defpositions}{\textbf{Основные положения, выносимые на~защиту:}}
\newcommand{\novelty}{\textbf{Научная новизна}}
\newcommand{\influence}{\textbf{Научная и практическая значимость}}
\newcommand{\reliability}{\textbf{Степень достоверности}}
\newcommand{\probation}{\textbf{Апробация работы.}}
\newcommand{\contribution}{\textbf{Личный вклад.}}
\newcommand{\publications}{\textbf{Публикации.}}

%{\actuality}
В настоящее время в Институте ядерной физики им. Г. И. Будкера СО РАН проводятся эксперименты на электрон-позитронном накопителе \text{ВЭПП-2000} \cite{VEPP} в диапозоне энергий от 320 МэВ до 2 ГэВ в системе центра масс. В местах встречи пучков установлены два детектора: криогенный магнитный детектор \text{КМД-3 \cite{CMD}} и сферический нейтральный детектор \text{СНД \cite{SND}.} Физическая программа экспериментов включает в себя измерение сечений электрон-позитронной аннигиляции в адроны на низких энергиях, а также измерение адронного вклада в вычисление аномального магнитного момента мюона {$(g-2)_\mu$}. Так как в данной области энергии расчеты в рамках пертурбативной квантовой хромодинамики невозможны, то основным источником изучения взаимодействия легких кварков являются данные, полученные в экспериментах. Набор экспериментальных данных с детектором \text{КМД-3} начат в декабре 2010 года. В апреле 2013 года эксперименты были остановлены в связи с модернизацией инжекционного комплекса \cite{status}. В конце 2016 года запущен новый инжекционный комплекс \text{ВЭПП-5 \cite{VEPP-5}} и возобновлен набор статистики детектором \text{КМД-3.}

 \aim\ данной работы является измерение сечения процесса {$e^+\:e^- \to K_{S}\:K_{L}\:\pi^0$}
 в области энергий от порога рождения \text{(1130 МэВ)} до \text{2 ГэВ.} В анализе процесса использовались данные, полученные в экспериментах 2011-2012 годов с детектором КМД-3. Интеграл светимости составил \text{$33.18$ пб$^{-1}$} в диапозоне энергий от \text{1.1 ГэВ} до \text{2 ГэВ. }
 %Процесс примечателен тем, что все три конечные частицы псевдоскалярные и имеют общую плоскость распада. Одно из главных физических свойства процесса - это нейтральность конечных частиц. (Ниже воткнуть.)
 %пересчитать точнее светимость для твоей области

В задачи данной работы входило:
\begin{enumerate}
  \item Выработать оптимальные критерии отбора событий для процесса {$e^+\:e^- \to K_{S}\:K_{L}\:\pi^0$}.
  \item Определить эффективность регистрации процесса с помощью Монте-Карло моделирования.
  \item Получить предварительное сечение процесса {$e^+\:e^- \to K_{S}\:K_{L}\:\pi^0$} в диапозоне энергий от порога рождения \text{(1130 МэВ)} до \text{2 ГэВ.}
\end{enumerate}

%\novelty\ данной работы заключается в исследовании малоизученного процесса  {e^+\:e^- \to $\itshape K_{S}\:K_{L}\:\pi^0$}, а так же измерение его вклада в поляризацию ваакума, используемое для вычисления аномального магнитного момента мюона {$(g-2)_\mu$}.

Сечение данного процесса до недавнего времени не было измерено, в настоящее время существуют пока что только предварительные результаты, представленные коллаборацией SND \cite{korneev}, и недавно опубликованная статья коллаборации BaBar \cite{BaBar}. %Результаты ранее сказанных коллабораций имеют различия в области энергий выше 1.8 ГэВ, таким образом, имеет место дополнительная задача --- подтвердить результаты одной из коллабораций.
%данной работы заключается в вычислении сечения процесса  {e^+\:e^- \to $\itshape K_{S}\:K_{L}\:\pi^0$} в настоящее время предварительно изммеренного коллаборациями BaBar и SND, а так же в измерении его вклада в поляризацию ваакума, используемое для вычисления аномального магнитного момента мюона {$(g-2)_\mu$}. %коряво

%Научная новизна заключается в том, что сечение данного процесса до сих пор не был измерен, в настоящее время существуют пока что только предварительные результаты, представленные коллаборациями Бабар и СНД.
%Сказать про физику здесь, изотопический спин, все частицы в одной плоскости, три псевдоскалярные частицы. % Характеристика работы по структуре во введении и в автореферате не отличается (ГОСТ Р 7.0.11, пункты 5.3.1 и 9.2.1), потому её загружаем из одного и того же внешнего файла, предварительно задав форму выделения некоторым параметрам

%% регистрируем счётчики в системе totcounter
\regtotcounter{totalcount@figure}
\regtotcounter{totalcount@table}       % Если поставить в преамбуле то ошибка в числе таблиц
\regtotcounter{TotPages}               % Если поставить в преамбуле то ошибка в числе страниц

Развитие источников синхротронного излучения (СИ), а именно магнито-оптических систем электронных накопительных колец \cite{bartolini_challenges_2021}, \cite{hettel_challenges_2014} источников СИ, дало возможность получать электронные пучки с малым эмиттансом, достигая дифракционного передела в широком диапазоне длин волн:
\begin{align}
	\epsilon_{x, y} < \sigma'_{r}\sigma_{r} = \lambda/4\pi,
	\label{eq:photons_emittance}
\end{align}
где $\lambda$ -- длина волны излучения, $\sigma'_{r}$ и $\sigma_{r}$ натуральная расходимость и размер излучения в перетяжке на источнике от электронного пучка с бесконечно малым эмиттансом. Дифракционно ограниченные источники обладают полной поперечной когерентностью, что выполняется в большинстве случаев для мягкого рентгеновского диапазона. По мере уменьшения длины волны степень поперечной когерентности излучения падает, излучение становится частично когерентным. 

Задача моделирования поля частично когерентного излучения является основной при проектировании оптических линий источников СИ последнего, четвёртого, поколения, так как именно случай с частично когерентным излучением реализуется в большинстве практических случаев и важен, например, для задач имиджинга. Метод трассировки лучей, реализованный, например, в коде SHADOW \cite{sanchez_del_rio_shadow3_2011}, являлся подходом, рутинно использовавшимся при проектировании источников синхротронного излучения третьего и второго поколений. Предпосылки для использования метода трассировки лучей основываются на низкой поперечной когерентности синхротронных источников излучения прошлых поколений. Однако, уже для источников третьего поколения дифракционный предел достигался в вертикальном направлении и предпосылки к использованию метода трассировки лучей становятся сомнительными. В целом, подходы трассировки лучей не дают удовлетворительную модель физических процессов, происходящих при генерации синхротронного излучения и не описывают процессы дифракции. Следует придерживаться правила применения трассировки лучей: характерные размеры особенностей оптики, должны быть много меньше размеров поперечной когерентности излучения на них. В ином случае необходимо принимать во внимание диффракционные эффекты.
%\footnote{Для шероховатостей нужно внести уточнение. Обратные размеры  зеркала ($1/L$) и длины волны ($1/\lambda$) определяют зону рассмотрение ошибок профиля зеркала. Обратная величина длины когерентности ($1/l_c$) задет условную границу, когда ошибки профиля зеркала рассматриваются как ошибки наклона для $k < 1/l_c$, где $k$ -- пространственные гармоники профиля зеркала, а когда должны быть рассмотрены как ошибки по высоте при $k > 1/l_c$. Можно скзаать, что ошибки по высоте, в целом, портят когерентность пучка излучения, а для излучения с итак плохой когерентностью учёт ошибок по высоте может быть опущен. Это опущение, естественно, происходит при проведении трассировки лучей, так как в этом методе не учитываются диффракционные эффекты.}
Для построения физической модели, соответствующей процессам генерации и распространения (пропагации) излучения, необходимо использовать подходы волновой оптики. При компьютерном моделировании под подходами волновой оптики подразумевается моделирование реальных электромагнитных полей, описываемых комплексными величинами и меняющихся при пропагации через оптические системы в соответствии с законами статистической оптики и Фурье оптики, \cite{goodman_introduction_2005}, \cite{goodman_statistical_2015}. Подходы волновой оптики позволяют учесть дифракционные эффекты для полностью когерентного излучения, однако моделирование частично когерентного синхротронного излучения остаётся сложной задачей. Один из походов в решении этой задачи реализован в коде Synchrotron Radiation Workshop (SRW) \cite{chubar_accurate_1998}, но этот подход облает рядом недостатков, которые будут разобраны в Главе~\ref{chapt2}.

В литературе даны все теоретически основы статистической природе синхротронного излучения. Необходимые материалы о свойствах источников синхротронного излучения третьего поколения можно найти, например в \cite{geloni_transverse_2008}. Однако, алгоритм, основанный на выводах работы, не был имплементирован в известные коды  моделирования синхротронного излучения. Описание процесса генерации синхротронного излучения, представленное в указанной работе, основывается на том, что дробовой шум в электронном пучке вызывает флуктуации электронной плотности, что в свою очередь привносит произвольные флуктуации амплитуды и фазы в распределение электромагнитного поля и является причиной характерной спайковой структуры одной реализации излучения. Под одной реализацией поля подразумевается идеально монохроматизированное\footnote{достаточна монохроматизация меньшая чем $1/\sigma_{T}$, где $\sigma_{T}$ длительность электронного пучка} поле после пролёта одного электронного пучка через вставное устройство источника СИ. Флуктуации электронной плотности меняются от пучка к пучку, и для получение характерного значение интенсивности поля, необходимо произвести усреднение по статистическому ансамблю электронных пучков. Метод основывается на прямом моделировании излучения каждого электрона (макроэлектрона)\footnote{Не представляется возможным моделировать излучения от каждого электрона отдельно, в виду того, что число электронов в пучке относится к большим числам. Поэтому электронный пучок приходится разбивать на кластеры, называемые в представленной работе макроэлектронами. Ограничение на минимально возможное число макроэлектронов приводится в тексте работы}, сложении полей от каждого из них и дальнейшем усреднении по статистическим реализациям. Для условности этот метод будет называться методом сложения амплитуд (МСА). В Главе~\ref{chapt2} этот подход также будет обсуждаться более подробно.

Два приведённых метода основываются, по сути, на прямом моделировании излучения от каждого электрона (макроэлектрона) и последующем суммировании. Расчёт такого поля весьма трудоёмок и занимает значительное время. В представленной работе предложен новый численно эффективный и отличающийся быстродействием алгоритм моделирования частично когерентного синхротронного излучения, основанный на ограничении пространственных гармоник комплексного гауссова шума огибающими поля. Для определённости новый метод будет называться метод СЕРВАЛ. Метод предлагает оценочный алгоритм расчёта частично когерентного поля ондуляторного излучения в практических случаях не уступающий методу сложения амплитуд. В Главе~\ref{chapt2} настоящей работы приводится соответствующий алгоритм и сравнительный анализ рассчитанных полей в дальней зоне, на источнике и функции взаимной когерентности, а также анализ границ применимости метода. В Главе~\ref{chapt3} приведены примеры использования СЕРВАЛа в трёх случаях: фокусировка при наличии конечной входной апертуры, классический двухщелевой эксперимент (интерферометр Юнга) и фокусировка рентгеновским зеркалом при наличии шероховатостей. Разработанный алгоритм стал частью среды OCELOT (https://github.com/ocelot-collab) для моделирования излучения лазеров на свободных электронах и синхротронных источников излучения. 

%По ходу работы был проведён эксперимент на European XFEL по регистрации спайковой структуры синхротронного излучения и измерении длины поперечной когерентности поля. Ондуляторная линия SASE2 на European XFEL работала в однопучковом режиме с одним закрытым ондулятором на ондуляторном резонансе первой гармоники 9.099 кэВ. Излучение пропускалось через двукристальный кремниевый монохроматор Si(333). Интенсивность излучения регистрировалась сцинтилляционным детектором. По проведении эксперимента, оказалось невозможно получить желаемый сигнал методами корреляционного анализа (расчёт функции взаимной когерентности второго порядка) из-за крайне низкого соотношения сигнал/шум, по всей видимости, близкого к 1. Однако, вопрос регистрации сигнала с приемлемым отношением сигнал/шум остаётся лишь вопросом улучшения светочувствительности детектора, что будет решено в будущем. Также было произведено моделирование ожидаемого сигнала, получена спайковая структура и рассчитана характерная длина когерентности излучения для модельного электронного пучка. Для подтверждения результата к модельному сигналу был добавлен шум детектора для того, чтобы подтвердить моделированием результат, полученный с реального сигнала. Результаты моделирования подтверждают полную потерю информации для функции взаимной когерентности уже при соотношении сигнал/шум равному четвёрке \rr{4 это по отношению к максимуму, надо рассчитать среднеквадратичное отклонение амплитуды сигнала, SNR будет меньше}.

\textbf{Целью} представленной работы являлась разработка и применение нового численно эффективного метода моделирования частично когерентного синхротронного излучения. Поставленная цель является \textbf{актуальными} для научного сообщества в связи с развитием источников синхротронного излучения и строительством новых источников четвёртого поколения. На данный момент имеется необходимость дальнейшего развития компьютерных кодов для моделирования излучения современных источников СИ. Научная \textbf{новизна} заключается в разработке уникального алгоритма расчёта частично когерентного поля. Предложенный алгоритм имеет \textbf{практическую ценность} и используется при проектировании рентгенооптических трактов источника синхротронного излучения четвёртого поколения ЦКП «СКИФ». 

%Так же ниже приводится список публикаций, в которых в той или иной степени были применены методы, развитые в представленной работе.\\

%\noindent\textbf{Список публикаций:}
%\\
%
%\noindent Trebushinin, A., Serkez, S., Veremchuk, M., Rakshun, Y. and Geloni, G., 
%Spatial-frequency features of radiation produced by a step-wise tapered undulator, (2021), \textit{J. Synchrotron Rad}, vol 28., part 3., https://doi.org/10.1107/S1600577521001958
%\\
%
%\noindent Serkez, S., Trebushinin, A., Veremchuk, M., and Geloni, G., Method for polarization shaping at free-electron lasers, Phys. Rev. Accel. Beams 22, 110705, https://doi.org/10.1103/PhysRevAccelBeams.22.110705




