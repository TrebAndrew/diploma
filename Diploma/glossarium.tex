\chapter*{Глоссарий}\label{glossarium}
\addcontentsline{toc}{chapter}{Глоссарий}	% Добавляем его в оглавление

\noindent \underline{Эмиттанс} релятивистского электронного пучка -- площадь фазового пространства в $x$, $x'$, $y$, $y'$ поперечных координатах, в работе не рассматривается понятие эмиттанса как объём шестимерного фазового пространства.

\noindent \underline{Пропагация} излучения -- распространение волнового фронта  вдоль оптической оси  от плоскость с позицией $z_1$ до позиции $z_2$.

\noindent \underline{$\text{Si}(\cdot)$} -- интегральный синус.

\noindent \underline{Комплексный гауссов шум} -- статистический процесс описываемы комплексным нормальным распределением: $Z = X + iY$, где $X$ и $Y$ нормальные распределения со средним $0$ и вариацией равной единице. В работе под комплексным гауссовым шумом подразумевается двумерный (или трехмерный) массив величин, где значение каждого элемента является комплексная случайная величина $Z$.

\noindent \underline{SERVAL} -- хищное млекопитающее семейства кошачьих, в работе используется как условное название для наименования предложенного метода моделирование частично когерентного поля ограничением пространственных гармоник комплексного гауссового шума огибающими поля. 

\noindent \underline{Макроэлектрон} -- понятие используемое при моделировании излучения электронного пучка, так как количество электронов в реальном электронном пучке велико и зачастую нет возможности моделировать на компьютере отдельно, приходится разбивать электронный пучок на кластеры -- макроэлектроны, число которых возможно моделировать на обычном персональном компьютере.

\noindent \underline{Метод сложения амплитуд} -- метод для расчёта излучения электронного пучка с конечным эмиттансом, основанный на сложении полей каждого макроэлектрона с последующим усреднением по статистическим реализациям.

\noindent \underline{Метод SRW}

\noindent \underline{Cпайковая структура синхротронного излучения}

\noindent \underline{SASE2}


\newpage
%============================================================================================================================
