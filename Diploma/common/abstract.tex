\chapter*{Аннотация }							% Заголовок
Представленная работа посвящена разработке алгоритма моделирования частично когерентного синхротронного излучения и его применения для расчёта нескольких рентгенооптических оптических схем. В работе рассматриваются теоретические основы процесса генерации и распространения частично когерентного синхротронного излучения, а также его статистические характеристики. По ходу изложения приводится описание двух известных методов моделирования частично когерентного излучения, первый из которых обсуждался в литературе, однако не был описан в программном коде и, соответственно, не применялся при расчёте рентгенооптических схем синхротронных источников излучения. С помощью этого метода в работе описан эффект влияния продольной когерентности излучения на его угловую расходимость. При этом второй метод широко используется, несмотря на то, что обладает рядом недостатков, обсуждаемых в работе. Оба метода дают физически достоверный результат для наблюдаемых интенсивностей синхротронного излучения, однако, только первый метод даёт физически правильные поля излучения. На основе проделанной работы по изучению статистических свойств синхротронного излучения, был создан алгоритм, называемый СЕРВАЛ, качественно отличающийся от уже известных подходов. Была проведена перекрёстная проверка СЕРВАЛа на совпадение генерируемого поля в источнике и дальней зоне, а также соответствующих функций поперечной когерентности. Применение СЕРВАЛ показано на примере трёх оптических схем: фокусировка при наличии конечной входной апертуры, классический двухщелевой эксперимент (интерферометр Юнга) и фокусировка рентгеновским зеркалом при наличии шероховатостей. В работе проанализированы границы применимости СЕРВАЛа и описано его быстродействие.